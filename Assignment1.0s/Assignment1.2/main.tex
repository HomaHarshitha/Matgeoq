\let\negmedspace\undefined
\let\negthickspace\undefined
\documentclass[journal]{IEEEtran}
\usepackage[a5paper, margin=10mm, onecolumn]{geometry}
%\usepackage{lmodern} % Ensure lmodern is loaded for pdflatex
\usepackage{tfrupee} % Include tfrupee package

\setlength{\headheight}{1cm} % Set the height of the header box
\setlength{\headsep}{0mm}     % Set the distance between the header box and the top of the text

\usepackage{gvv-book}
\usepackage{gvv}
\usepackage{cite}
\usepackage{amsmath,amssymb,amsfonts,amsthm}
\usepackage{algorithmic}
\usepackage{graphicx}
\usepackage{textcomp}
\usepackage{xcolor}
\usepackage{txfonts}
\usepackage{listings}
\usepackage{enumitem}
\usepackage{mathtools}
\usepackage{gensymb}
\usepackage{comment}
\usepackage[breaklinks=true]{hyperref}
\usepackage{tkz-euclide} 
\usepackage{listings}
% \usepackage{gvv}                                        
\def\inputGnumericTable{}                                 
\usepackage[latin1]{inputenc}                                
\usepackage{color}                                            
\usepackage{array}                                            
\usepackage{longtable}                                       
\usepackage{calc}                                             
\usepackage{multirow}                                         
\usepackage{hhline}                                           
\usepackage{ifthen}                                           
\usepackage{lscape}
\begin{document}

\bibliographystyle{IEEEtran}
\vspace{3cm}

\title{1-1.5-21}
\author{EE24BTECH11062 - Homa Harshitha Vuddanti
}
% \maketitle
% \newpage
% \bigskip
{\let\newpage\relax\maketitle}

\renewcommand{\thefigure}{\theenumi}
\renewcommand{\thetable}{\theenumi}
\setlength{\intextsep}{10pt} % Space between text and floats


\numberwithin{equation}{enumi}
\numberwithin{figure}{enumi}
\renewcommand{\thetable}{\theenumi}


\textbf{Question}:\\
Find the ratio in which $\vec{P}=\myvec{4\\m}$ divides the line segment joining the points $\vec{A}=\myvec{2\\3}$ and $\vec{B}=\myvec{6\\-3}$. Hence find $m$.
\\
\textbf{Solution: }\\
Given,\begin{table}[h!]    
  \centering
  \begin{tabular}[12pt]{ |c| c|}
    \hline
    \textbf{Variable} & \textbf{Description}\\ 
    \hline
    $k$ & Ratio in which P divides line AB. \\
    \hline 
    $m$ & $y$-coordinate of point P\\
    \hline
    \end{tabular}

  \caption{Variables Used}
  \label{1-1.5-21-table}
\end{table}
 \begin{align}
\vec{A}=\myvec{2\\3}\label{1-1.5-21-1}\\
\vec{B}=\myvec{6\\-3}\label{1-1.5-21-2}\\
\vec{P}=\myvec{4\\m}\label{1-1.5-21-3}
\end{align}
\text{Section formula:}\\
If C divides AB in the ratio $k$:1; 
\begin{align}
C&= \frac{kB+A}{k+1}\label{1-1.5-21-4}
\end{align}
\begin{align}
\myvec{4\\m}&=\frac{k\myvec{6\\-3}+\myvec{2\\3}}{k+1}\label{1-1.5-21-5}\\
\myvec{4\\m}&=\frac{1}{k+1}\myvec{6k+2\\-3k+3}\label{1-1.5-21-6}\\
6k+2&=4\brak{k+1}\label{1-1.5-21-7}\\
-3k+3&=m\brak{k+1}\label{1-1.5-21-8}
\end{align}
   From equations \eqref{1-1.5-21-7} and \eqref{1-1.5-21-8}, 
\begin{align}
k&=1,\\
m&=0
\end{align}
Hence, $ratio=1:1;\\
m=0$.
\begin{figure}[h!]
   \centering
   \includegraphics[width=1\linewidth]{fig/Fig.png}
   \caption{Plot}
   \label{1-1.5-21-fig-1}
\end{figure}
\end{document}  
\end{document}



